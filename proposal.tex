\documentclass[apa6]{article}
\usepackage{cite}

%opening
\title{User Specification in Bayesian Modeling: A Proposal}
\author{Zachary J. Roman}

\begin{document}

\maketitle

\begin{abstract}
	Bayesian MCMC analysis have become popular over the past decade. This is largely do to technological advances which has allowed researchers to conduct these analysis in a reasonable time frame. In early MCMC implementations it was recommended to use 2 chains and 2000 chain iterations (1000 burn in). However, this was due to computational efficiency at the time. More recently recommendations of 4 chains is common \cite{plummer2006coda}. While it is accepted that more chains and more chain iterations provides more confidence, computational restrictions make this a debate. The current study aims to explore the convergence of varying MCMC chains and chain iterations to explore the trade off of time and confidence.
	
	Further, it is important to identify certain situations where convergence and confidence vary. One such situation is how the prior distribution is specified (diffuse, strong correct, strong incorrect, weak correct, weak incorrect). Thus the current study aims to explore convergence of parameter estimates under the differing conditions of: Number of chains, number of chain iterations, and prior specification.


\end{abstract}

\section{Introduction}

In the Bayesian Monte Carlo Markov Chain (MCMC) world, the researcher has a large weight on their shoulders. Pre-specification of elements outside the model are a necessary part of the analytic process. In a simple regression model ($Y = X\beta_{1} + \beta_{0} + \epsilon$) for example, the researcher must specify: Prior distributions for both $\beta$ coefficients(most likely Cauchey), choose a number of MCMC chains (usually 2 to 4), the length of the chains (usually between 2,000 and 10,000 iterations), and proportion of chain length dedicated to burn in (Usually between 10 to 50 percent). There is a large body of literature on choices regarding priors, however I'm hard pressed to find analytic rules regarding the number of chains, chain length, and burn in proportion. 

\subsection{Problem}

If you use the Bayesian modeling software OpenBUGS \cite{spiegelhalter2007openbugs}, you will notice the default chain number is 2. If you use 

\bibliography{mybib.bib}{}
\bibliographystyle{plain}

\end{document}
